% Created 2019-04-25 四 21:24
% Intended LaTeX compiler: pdflatex
\documentclass[11pt]{article}
\usepackage[utf8]{inputenc}
\usepackage[T1]{fontenc}
\usepackage{ctex}
\usepackage{graphicx}
\usepackage{grffile}
\usepackage{longtable}
\usepackage{wrapfig}
\usepackage{rotating}
\usepackage[normalem]{ulem}
\usepackage{amsmath}
\usepackage{textcomp}
\usepackage{amssymb}
\usepackage{capt-of}
\usepackage{hyperref}
\author{dx}
\date{\textit{<2019-04-24 三>}}
\title{Feynman Path Integral Note}
\hypersetup{
 pdfauthor={dx},
 pdftitle={Feynman Path Integral Note},
 pdfkeywords={},
 pdfsubject={},
 pdfcreator={Emacs 26.1 (Org mode 9.1.9)}, 
 pdflang={English}}
\begin{document}

\maketitle
\tableofcontents


\section{Propagators}
\label{sec:org3aa55a3}

\subsection{Heisenberg 表象中的态的内积}
\label{sec:org0b97df2}

Propagator 可以看作是 Heisenberg 表象中的态的内积, 即
\begin{align}
  \langle x_N, t_N | x_0 , t_0 \rangle
  = [\langle x_N | U(t_N, 0)]
   \cdot[ U(0,t_0)| x_0 \rangle]
\end{align}

\subsection{演化算符的矩阵元}
\label{sec:org25150d6}

Propagator 可以看作是演化算符的矩阵元, 即
\begin{align}
  \langle x_N, t_N | x_0 , t_0 \rangle
  = \langle x_N | U(t_N, t_0) | x_0 \rangle
\end{align}
物理意义解释为: \(t_0\) 时刻处于 \(|x_0\rangle\) , 然后演化到 \(t_N\) 时刻,
这时在态 \(| x_N\rangle\) 上的投影, 即在态 \(| x_N\rangle\) 上的概率振幅,
就是 Propagator .

\section{Feynman's Path Interal}
\label{sec:orgea5d2bf}

\subsection{核心思路}
\label{sec:org198f3e3}

算得了 Propagator , 就知道了体系的演化, 问题就得到了解决. 所以是目标就
是计算 Propagator .

一个基本的假设是
\begin{align}
    \langle x_N, t_N | x_0 , t_0 \rangle \quad \mathrm{corresponds\quad to }
  \quad e^{\frac{\mathrm{i}}{\hbar}S(0,N)}
\end{align}
其中 \(S\) 是作用量
\begin{align}
  S(N,0) =
   \int_{t_0}^{t_N}\mathrm{d}t \cdot L(x, \dot{x}) 
\end{align}
\(L\) 是经典的拉氏量. 但是 \(L\) 是 \(x\) 和 \(\dot{x}\) 的函数. 所以只有选定
一个路径后, \(S(N,0)\) 才有明确定义. 怎么办呢?

解决方法是, 插入完备基. 在 Heisenberg 表象中, 每个时刻的位置的本征态都
可以看作是一组完备基. 即
\begin{align}
  \int \mathrm{d}^3 x\cdot |\vec{x},t\rangle \langle \vec{x},t |
  = 1
\end{align}
在 \(t_0\) 和 \(t_N\) 之间插入无穷多组完备基, 也就是
\begin{align*}
  &\langle x_N, t_N | x_0 , t_0 \rangle \\
   =&
   \int\mathrm{d}^3 x_{N-1}\cdots
   \int\mathrm{d}^3 x_1 \cdot
   \langle x_N, t_N |\vec{x}_{N-1},t_{N-1}\rangle
   \langle \vec{x}_{N-1},t_{N-1} |
   \cdots |\vec{x}_1, t_1\rangle
   \langle \vec{x}_1, t_1 | x_0 , t_0 \rangle
\end{align*}
其中 \(N\to \infty\) . 这样相当于对于所有的路径都作了计算, 也就是
\begin{align}
      \langle x_N, t_N | x_0 , t_0 \rangle \quad \sim
  \quad \sum_{\mathrm{all\,paths}} e^{\frac{\mathrm{i}}{\hbar}S(N,0)}
\end{align}
对于无穷短间隔内的 Propagator , 将其假设为
\begin{align}
  \langle x +\Delta x , t + \Delta t |x , t \rangle
  = \frac{1}{\omega(\Delta t)}
   e^{\frac{\mathrm{i}}{\hbar}S}
\end{align}
其中 \(\omega(\Delta t)\) 是只与 \(\Delta t\) 的大小有关的一个归一化系数.

这样, 就得到了 Feynman 的 Path Integral 的表达式
\begin{align*}
  &\langle x_N, t_N | x_0 , t_0 \rangle \\
  =& \lim_{N\to \infty}\frac{1}{[\omega(\Delta t)]^{N-1}}\cdot
   \int\mathrm{d}^3 x_{N-1}\cdots
   \int\mathrm{d}^3 x_1 \cdot e^{\frac{\mathrm{i}}{\hbar}S(N,0)}\\
  =&\int_{x_1}^{x_N}\mathcal{D}[x(t)] \cdot e^{\frac{\mathrm{i}}{\hbar}S(N,0)}\\
\end{align*}
第二个等号采用了新的记号
\begin{align}
  \int_{x_1}^{x_N} \mathcal{D}[x(t)]\equiv \lim_{N\to \infty}\frac{1}{[\omega(\Delta t)]^{N-1}}\cdot
   \int\mathrm{d}^3 x_{N-1}\cdots
   \int\mathrm{d}^3 x_1 
\end{align}

\subsection{归一化系数}
\label{sec:orgf49edaa}

下面来求归一化因子 \(\omega(\Delta t)\) (假设它与势无关), 可以用自由粒子
的拉氏量. 然后, 利用完备基的正交归一性
\begin{align}
  \lim_{\Delta t\to 0}   \langle x +\Delta x , t + \Delta t |x , t \rangle
  = \delta (\Delta x)
\end{align}
那么
\begin{align}
  \lim_{\Delta t\to 0}\frac{1}{\omega(\Delta t)}
   e^{\frac{\mathrm{i}}{\hbar}S} = \delta (\Delta x)
\end{align}
下面先把 \(S\) 算出来
\begin{align}
  S = \int_t^{t+\Delta t} \mathrm{d} t' \cdot \left[
  \frac{m}{2}\left(\frac{\Delta x}{\Delta t}\right)^2  \right]
  = \frac{m (\Delta x) ^2}{2\Delta t}
\end{align}
上式中由于是自由粒子, 所以粒子是直线运动, 速度为 \(\Delta x/\Delta t\) .
将算出来的 \(S\) 代回原式并对 \(\Delta x\) 在全空间积分得
\begin{align}
  \frac{1}{\omega(\Delta t)} \cdot\int \mathrm{d}\Delta x \cdot
  e^{\frac{\mathrm{im (\Delta x)^2}}{2 \hbar \Delta t}}
  =\frac{1}{\omega(\Delta t)} \sqrt{\pi\frac{2\hbar \Delta t}{m}}
  =1
\end{align}
即求得归一化系数为
\begin{align}
  \frac{1}{\omega (\Delta t)} = \sqrt{\frac{m}{2\pi \hbar\Delta t}}
\end{align}

\section{Classical Limit}
\label{sec:org49ff6b9}

当 \(\hbar \to 0\) 时, 只有经典轨道有贡献. 所谓经典轨道, 即满足 Hamilton
原理的路径
\begin{align}
  \delta S = 0
\end{align}
. 这是因为, 不满足 Hamilton 原理的作用量 \(\delta S \neq 0\) , 而 \(\hbar
\to 0\) , 所以作用量即使有很小的改变, 它对应的相位也会有很大的改变, 那么它在对
所有路径求和时, 相位会相干相消. 而只有满足 Hamilton 原理的路径, 它有微
小的改变时, 它的相位不会相消. 作用量在 \([-\hbar \pi, \hbar\pi]\) 内都会
有贡献.

一个周期内的相位都会相消, 只有作用量取平稳值的时候没有其它相位与它相消.

\section{Equivalence to Schrodinger's Wave Mechanics}
\label{sec:org8eb0bc7}

考虑从 Propagator 对时间的微分入手, 将从 \(t_0, x_0\) 到任意时刻 \(t, x\) 的
Propagator 在 \(t-\Delta t\) 处展开
\begin{align}
  \langle x, t| x_0, t_0 \rangle = \langle x, t-\Delta t | x_0 ,t_0 \rangle
  + \Delta t \frac{\partial}{\partial t}
  \langle x, t-\Delta t | x_0 ,t_0 \rangle + \mathcal{O}[(\Delta t)^2]
\end{align}
同时, 也可以用 Path Integral 的方法计算 Propagator 然后再对应 \(\Delta
t\) 的一阶项. 这样就用 Path Integral 求得 Propagator 对时间的一阶导数.

可以考虑在离 \(|x, t\rangle\) 无穷近的地方插入一组完备基
\begin{align}
  \langle x, t| x_0, t_0 \rangle = \sqrt{\frac{\lambda}{\pi}}
  \int_{-\infty}^{+\infty} \mathrm{d}\Delta x \cdot e^{-\lambda (\Delta x)^2 }
  \cdot e^{-\frac{\mathrm{i} V \Delta t}{\hbar}} \cdot
  \langle x- \Delta x, t- \Delta t| x_0, t_0 \rangle
\end{align}
其中拉氏量已代为 \(L = \frac{m}{2}\left( \frac{\Delta x}{ \Delta t}
\right)^2 - V\) , 为了简便记 \(\lambda = \frac{m}{2\mathrm{i}\hbar
\Delta t}\sim \frac{1}{\Delta t}\) . 
将上式最后一项在 \(\langle x, t- \Delta t |\) 附近展开
\begin{align*}
  \langle x- \Delta x, t- \Delta t| x_0, t_0 \rangle
  =& \langle x, t- \Delta t| x_0, t_0 \rangle + \mathrm{Linear\, term}\\
  &+ \frac{(\Delta x)^2}{2}\cdot \frac{\partial^2}{\partial x^2}
  \langle x, t- \Delta t| x_0, t_0 \rangle + \mathcal{O}[(\Delta x)^3]
\end{align*}
上式中的线性项及其它奇数次幂项由于是奇函数, 积分后都为零, 所以不加考虑.

代回积分式中, 并将 \(e^{-\frac{\mathrm{i} V\Delta t}{\hbar}}\) 也按
\(\Delta t\) 进行 Taylor 展开, 可得
\begin{align*}
  \langle x, t| x_0, t_0 \rangle =  \langle x, t-\Delta t | x_0 ,t_0 \rangle
  + \left[\frac{1}{2}\frac{\partial^2}{\partial x^2} \cdot \frac{1}{2}\frac{1}{\lambda}
   -\frac{\mathrm{i}}{\hbar}V \Delta t\right] \langle x, t- \Delta t| x_0, t_0 \rangle
  + \mathcal{O}[(\Delta t)^3]
\end{align*}
注意式其中 \(\frac{1}{\lambda}\sim t\) . 其中利用了积分公式
\begin{align}
  \int_{-\infty}^{+\infty}\mathrm{d} x\cdot e^{-\lambda x^2}
  =& \sqrt{\frac{\pi}{\lambda}} \\
  \int_{-\infty}^{+\infty}\mathrm{d} x\cdot x^2e^{-\lambda x^2}
  =& -\frac{\partial}{\partial \lambda}\sqrt{\frac{\pi}{\lambda}} 
  = \frac{1}{2}\sqrt{\frac{\pi}{\lambda^3}}
\end{align}
比较两种展开的一阶项可得
\begin{align}
  \mathrm{i}\hbar \frac{\partial}{\partial t}\langle x, t- \Delta t| x_0, t_0 \rangle
  =\left[ -\frac{\hbar^2}{2m}\frac{\partial^2}{\partial x^2}
  +V\right]\langle x, t- \Delta t| x_0, t_0 \rangle
\end{align}
此即 Schrodinger Equation !

\section{Reference}
\label{sec:org2b5d1e2}

J. J Sakurai, Jim Napolitano, Modern Quantum Mechanics 2ed:
\begin{itemize}
\item Chap 2.6 Propagators and Feynman Path Integral
\end{itemize}

R. Shankar, Principles of Quantum Mechanics 2ed:
\begin{itemize}
\item Chap 8 The Path Integral Formulation of Quantum Theory
\end{itemize}
\end{document}